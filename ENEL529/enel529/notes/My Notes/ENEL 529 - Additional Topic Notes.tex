

%% ENEL 529 - Additional Topic Notes
%% =================================
\title{ENEL 529 - ADDITIONAL TOPICS}
\author{Kyle Derby MacInnis}
\date{\today}

\begin{document}
\maketitle


\section*{ADDITIONAL TOPIC - FOURIER TRANSFORM}

 \subsection{SIGNAL TYPES}
 	There are two kinds of signals:
 	\begin{enumerate}
 	\item Deterministic
 	\item Random
 	\end{enumerate}

 	\subsubsection{Deterministic Signals}
	 	Deterministic Signals are signals in which the value of the signal is predetermined, and no uncertainties exist about what value it may have at any time.

 		Deterministic Signals come in two flavours:
 		\begin{enumerate}
 		\item Aperiodic Signals
 		\item Periodic Signals
 		\end{enumerate}

 		\subsubsubsection{Aperiodic Signals}
	 		Aperiodic Signals, also known as Nonperiodic signals, do not have a cyclical nature about them, are of the form:

 			\[E = \int_{-\infty}^{\infty}x^2(t) dt\]

 			These Signals, are sometimes called "Energy Signals". They are so-called because they have a non-zero finite Energy, $E$.

 		\subsubsubsection{Periodic Signals}
 			Periodic Signals, are signals for which there exists a constant $T$ such that:

 			\[x(t) = x(t + T), \indent -\infty \le t \le \infty\]

 			These Signals, are sometimes called "Power Signals". They are called power signals, if, and only if (iff), the signal has finite, but nonzero power, $P$ given by:


 			\[E = \lim_{T\rightarrow\infty}\frac{1}{T}\int_{-T/2}^{T/2}x^2(t)dt\]


 		Note that a Power Signal has Finite Power, and Infinite Energy, whilst an Energy Signal has zero average power, and finite Energy.

 			
\end{document}